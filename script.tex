\documentclass{article}
\usepackage[utf8]{inputenc}
\usepackage{dirtytalk}
\usepackage{color}
\usepackage[acronym]{glossaries}
\usepackage[backend=biber,style=alphabetic,citestyle=authoryear]{biblatex}
\addbibresource{bib.bib}
\title{Responsibility Fap}
\author{Mischa}


\makeglossaries

\newacronym{mrinfes}{mRINFES}{Moral Responsibility is Important and Necessary for a Functional and Ethical Society}
\newacronym{cr}{CR}{Control Requirement}
\newacronym{ai}{AI}{Artificial Intelligence}
\newacronym{ml}{ML}{Machine Learning}
\newacronym{aws}{AWS}{Autonomous Weapon System}



%Numbered environment
\newcounter{example}[section]
\newenvironment{example}[1][]{\refstepcounter{example}\par\medskip
   \noindent \textbf{Example~\theexample. #1} \rmfamily}{\medskip}


\begin{document}
\begin{titlepage}
	\maketitle
\end{titlepage}
\tableofcontents
\newpage
\section{Introduction}

Nowadays it is an obivious statement to make that we live in a time in which
technology is ubiquitous and new technology is being developed at an
unprecidented rate. It penetrates our society and is one of the adhesives that
hold 'the system' in place. We must only envision a world in which cars do not
exist; or refrigerators; or the internet; or x-ray machines to see how much of
our (everyday) lives depends and is shaped by it. I also don't think that I go
out on a limb when I say that we integrate some technology relatively fast into
our lives. Iphone in 2007 and a couple years later everybody has a smartphone
bla bla bli blub


Traffic lights are installed and programmed, perhaps superficially supervised
and occasionally maintained and updated. peepee

In this work I will discuss technology and moral responsibility and how the two
relate to each other. Specifically, I will investigate the different ways we can seek
for moral responisbility in situations where an autonomous machine is involved.

Put more here.

But first, let us examine the traditional way of how we ascribe moral
responisbility in situations where technology is involved:

Suppose the following situations: A person hits another person with a hammer and
kills them. A newly installed dam breaks and a city is flooded. A hacker manages
to get access to a digital banking system through his own computer and steals a
good deal of money.\\

The hammer, the dam and the hacker's computer are technology that is directly
involved in morally critical situations. Yet, we abstain from blaming these
artifacts for what has happened in the respective situations. We also do not put
the events off as natural tragedies, as we do when a storm destroys a house or
an avalanche kills a skier in the mountains. We naturally ascribe the
responsibility for the events to the people behind the technology.
The person who wielded the hammer, the architect of the dam, the hacker.
These people used the technology as a tool to achive their own end and they are
responsible for the effects, that the technology has on our world, whether they
achieve the end or not (in the case of the dam-architect).
To view technology as tools or instruments used by humans and the humans as the ultimately
responsible entities for the technology is what Heidegger calls the \textit{instrumentalist
definition of technology} (\cite{heidegger1977technology}). 

\say{We ask the question concerning technology when we ask what it is. Everyone
knows the two statements that answer our question. One says: Technology is a means to an end. The
other says: Technology is a human
activity.}(\cite[p.4]{heidegger1977technology})

Thus, according to the instrumentalist definition, technology is something that
is intimately connected to humans and inherits its moral standing from the
quality of the \textit{human} end that it tries to achieve and from the way
it is used by \textit{humans}. Technology is not moral on its own. Only in the
context of human ends and actions. If it is humans that decide to use technlogy
for some end it is naturally them, who are responsible for the results.\\

In the advent of machine learning we are facing a type of technology that is
intentinally becoming more and more autonomous, making decisions on its own
without any human supervision, without any human being able to predict the
decisions and even without any human being able to explain the decisions. The
Machines are essentially black boxes making decisions and affecting the world in
morally significant ways:

Autonomous Vehicles are being developed to populate the streets and naviagte
dangerous situations.
Machine Learning Algorithms analyse our behaviour on the internet, recommend
content that they find we would be interested in and influence us in this
manner.
There is a multitude of applications for ML in health care. We can easily
imagine that medical practitioners will increasingly rely on tools that diagnose
diseases and even propose treatments. Eventually, even patients might even cut
out the middle man and receive their medical care from artificial physicians.
There already are services that provide a kind of psychotherapy by texting with
a chatbot. [zitieren]
Autonomous Weapon Systems (\acrshort{aws}) are being developed. The aim is to
create war robots that can be sent into the battle field and they would be able
to decide on their own whether to kill a target or not.



%Instrumental Theory:

%(\cite{johnson2006computer}): Computers are moral entites but not moral agents
%For most of the time human technology has been used as a tool.
%The user or manufacturer has the responisibility for what the tool does.
%This has been working so far quite well and the instrumental theory was a valid
%and accepted extension of our moral understanding.
%We are now in a time, where new technologies, that exhibit more and more
%autonomy challenge this stance, that they are mere tools. How should we handle
%des question of responisibility with this new technology? Is there are
%responisibility gap?
%
%Introduction of new technologies: ML

\subsection{The Dirty Problem}

According to Matthias (\cite{Matthias_2004}) the reason why we can hold either the
manufacturer or the operator of a machine responsible for what it effects in the
world, is because we can sensibly say that they are the moral agents who were in
control of said machine. Matthias claims that responsibility implies control:

\vspace{.8em}
\say{[An] agent can be considered responsible only if he knows the particular
facts surrounding his action, and if he is able to freely form a decision to
act, and to select one of a suitable set of available alternative actions based
on these facts.} (\cite[p.175]{Matthias_2004})

\vspace{.8em}
This means that we can only hold someone responsible for something they have
done if they had sufficient control over their action.
This is widely refered to as the Control Requirement (\acrshort{cr}) [zitieren].

Converesly, if an agent does not have sufficient control, we can acsribe at most
partial responsibility, if any, to them.\\
This notion of control complements the intrumental theory nicely: The
manufacturer and operator have control over their machines, thus they are the
ones who are responsible if something happens because of the machines.
It is then very clear how to assign responisbility in critical situations: If the operator uses the
machine in accordance with the manufactureres specifications and something goes
wrong, we say that the manufacturer is responsible. If the operator deviates
from the manufacturers specifications and something goes wrong, we say the
operator is responsible(\cite[p.175]{Matthias_2004})

Enter Machine Learning:
We are now in a time where computer scientists and engineers work on increasing
the autonomy of their programs and machines by using techniques that can be
bundeled by the term \textit{machine learning} (\acrshort{ml}). `Autonomy' in this sense
means that we allow the technology to make it's own decisions based on it's
prior experience without the programmer or user knowing what the decision is
going to be. 

%Put here: Machines can play Go, AWS, medical exper systems, even the youtube
%algorithm

This leads to a change in the classical roles of the manufacturer
(programmer) and operator (user) insofar that the amount of control they exert
over the machines diminishes as the machines become more and more autonomous.
Arguably, our traditional ways of ascribing responsibility are challenged. If we
agree with the CR and 

%Current practices in Artificial Intelligence (\acrshort{ai}) challenge this
%stance. The classical roles of manufacturer (programmer) and operator (user)
%change insofar that the amount of control they exert over the machines
%diminishes as the machines become increasingly autonomous.

Current practices in Artificial Intelligence (\acrshort{ai}) aim at increasing
the autonomy of machines and their software [zitieren]. `Autonomy' in this sense
means that we allow the technology to make it's own decisions based on it's
prior experience without the programmer or user knowing what the decision is
going to be. In other words, they reduce the amount of control the programmer or
user has over the machines. This is primarily the case for technology that uses
machine learning to acquire a certain behaviour. [Passt hier eine beschreibung
von machine learning rein?] At the same time the actions of
such technologies have more and more impact on the people surrounding them
[zitieren].\\

The real question:
Now here comes the question of this work: Who can sensibly be held responsible for the
actions of an autonomous machine?
Who is responsible if a driverless car runs over a person? Who is responsible if
an artificial physician proposes a wrong treatment for a patient? Who is
responisble for a war crime commited by an autonomous weapon system
(\acrshort{aws})?\footnote{Talk about asymetry of blame and reward. Refer to
later in next section perhaps, because here I mention responisbility only in the
sense of blaming someone}\\

Matthias says that our current practices of ascribing moral responsibility
fail at finding an appropriate target when an autonomous machine is
strongly involved in a situation. He calls this problem \textit{the
responsibility gap}.\\
%are not designed for dealing with autonomous machines and we are, thus, facing
%a responsibility gap. \\

The considerations above, do not entail that our current practices of
dealing with the effects machines have on our world must necessarily change,
but rather that the contemporary and foreseeable developments in AI and ML
challenge our current practices and motivate their reevaluation.

On the following pages I will first give descriptions of various accounts for
moral responsiblity and will then proceed to describing and debating the
different approaches philosophers have proposed for dealing with the assumed
responisbility gap.

%BIS HIER
%
%The idea that manufacturers and users lose control over an autonomous machine
%does not only stem from the theoretical definition of the word 'autonomous'. As
%Matthias argues, the loss of control can be found in the very way we put
%ML-techniques into practice (\cite[p.181-182]{Matthias_2004}). 
%
%Russell and Norvig define an agent as learning 
%
%\vspace{.8em}
%\say{if it improves its performance on future taskes after making observations
%about the real world.}(\cite[p.693]{russell2010artificial})
%
%\vspace{.8em}
%
%Technology that is being developed today often has the aim to become more and
%more autonomous. Autonomous in this sense means that we allow the technology to
%make it's own decisions based on it's prior experience without the programmer or
%user knowing what the decision is going to be. This is primarily the case for
%technology that uses machine learning to acquire a certain behaviour.
%As Matthias (\cite{Matthias_2004}) points out, the growing autonomy challenges our
%existing pracices for ascription of moral responsibility. Matthias says that
%in order to be responsible for something, one requires to be in control of it.
%This widely refered to as the Control Requirement (\acrshort{cr}) [zitieren].
%It follows, that neither the programmer, nor the user of an autonomous machine,
%can sensibly be said to be responsible for the actions of that machine.
\newpage
\section{What is Moral Responsibility}

Before jumping into any analysis of the responsibility gap as described above,
it makes sense to first explore what I mean, when I speak of moral
responsibilty. While we generally have an intuitive understanding of what we mean
by (moral) responsibility, it is important for the following discussion to have
a more rigorous account of the term. PUT SOMETHING HERE.
ANGELA SMITH DESCRIBES THE DIFFERENT MEANINGS OF "A HOLDS B RESPONSIBLE FOR X"

\subsection{Strawson}

There is an ongoing discussion in philosophy about the existance of determinism
and it's impact on free will, and some philosophers deem free will to be closely related to
moral responsibility [zitieren]. To discuss this topic is notbla bla nor is it
my intention to take a position on this issue. Instead I will try to elegantly
sidestep the matter by taking a Strawsonian approach on moral responsibility.

In ``Freedom and Resentment'' P.F.Strawson gives and account of our moral
practices and tries to explain the mechanisms behind them. These mechanisms lay
the groundwork for what can be understood as moral responsibility.
In the centre of Strawson's argumentation lies \say{the very great importance that
we attach to the attitudes and intetions towards us of other human beings
[...]}(\cite[p.5]{Strawson1962}). In other words, we care a lot about how other
people treat us. We like it, if other people treat us with what we interpret as
respect and goodwill and we do not like it, if other people treat us with what
we interpret as illwill or indifference. Depending on how other people treat us
and which attitudes we ascribe to them, we in turn develop and adjust our own
attitudes towards them. Strawson calls the attitudes we form as a reaction to
other people's attitudes towards us (quite fittingly) our \textit{reactive
attitudes}. Examples for such attitudes are resentment, indignation, gratitude.
These reactive attitudes form the basis for our practices of blaming and
praising other people.

BEISPIEL EINFÜGEN!!


\begin{example}
	Matt is seventeen and likes playing computer games. His ten year old brother
	Charly often watches him play and frequently asks Matt, if he can play
	too. Matt usually denies Charly's request. Charly finds this unfair
	because Matt can play so much and he can only watch. Charly develops
	slight resentment against his brother because in his eyes, Matt
	does not care enough about him to fulfill Charly's wish of playing.
	Eventually, Charly runs to his mother and complains about Matt's
	unwillingness to allow Charly to play on the computer.
\end{example}

The primitive example above portrays the mechanism, Strawson tries to describe.
In the situation Charly interprets that his brother, Matt, treats him with an
attitude he does not like: indifference. This prompts Charly to develop
resentment towards Matt as a reactive attitude. Charly's going to his mother and
complaining about Matt is his way of blaming Matt.

Strawson stresses the importance of attitude behind an action, for we evaluate
other people and their actions strongly on the basis of their attitudes and
intentions. The same action with different attitudes elicits different reactions
from us. Strawson gives the example of someone stepping on his hand. If P-Boy
found that they did it accidentally and they were sorry for injuring him, he
would feel the pain in his hand, but probably no (appropriate) resentment
towards them. If, on the other hand, he found that they stepped on his hand out
of malevolence or were indifferent to what had happened, Strawsons reaction
would include some kind of resentment towards the other person. The same is
true, for when another person benefits us in some way. The degree of gratitude
we would feel towards them would differ, depending on whether they did it on
purpose and out of good will or accidentally (\cite[p.6]{Strawson1962}).

I should also point out, just like Strawson repeatedly does (\cite[p.5,
p.7]{Strawson1962}), that the way reactive attitudes work is much more
complicated than can be explained in this text. There is a complex
interplay between different parties and the attitudes vary on a broad spectrum
as well as in intensity.\\

The type of reactive attitudes I have described until now is generally about close
personal interactions with other people. They develop because of the way other
people treat specifically \textit{us}. 
However, reactive attitudes are not only a personal phenomenon but are also developed
and affected by how the objects of these attitudes treat other people.
Thus, Strawson introduces another class of reactive attitudes, which he calls
\textit{vicarious} or \textit{impersonal} reactive attitudes. These attitudes
target the behaviour or will of others independent of who is affected by them.
To be clear: These impersonal reactive attitudes can also be
developed if \textit{we} are the suffering party, but \say{[...] they are
essentially capable of being vicarious}(\cite[p.15]{Strawson1962})
THE 'TO BE CLEAR' IS NOT AS CLEAR AS IT SHOULD BE.

Strawson proceeds and gives these vicarious reactive attitudes the
qualifier \textit{'moral'} and the objects of such reactive attitudes are said
to have done something that has moral value (positive or negative) to us
(\cite[p.15]{Strawson1962}). And thus, Strawson has linked the concept of
moralitiy with his reactive attitudes.


\begin{example}
	Clara likes to read the newspaper in the morning. Today, she finds an
	article about a CEO of a big international company and how he knowingly
	choses suppliers that violate human rights to drive the price of their
	commodities down. Clara does not like this behaviour.
\end{example}


It is clear that Clara is not personally affected (at least not directly) by the
behaviour of the CEO. She still develops a reactive attitude towards him on the
basis of his indifference regarding humabn rights and the people who suffer
because of it. What Clara experiences is moral indignation.

To sum it all up: According to Strawson, we expect from other people that they
behave in accordance with attitudes of respect and goodwill. Depending on
whether they cohere with these expectations, we exhibit resentment or gratitude
(reactive attitudes) towards them. We blame or praise other people on the basis
of these reactive attitudes. Morality comes into play, when we acknowlidge that
we expect certain behaviour not only towars us, \say{[...] but towards all those on
whose behalf moral indignation may be felt [...]} (\cite[p.16]{Strawson1962}).

In light of this account, moral responsibility is not a metaphysical entity.
From a Strawsonian perspective, to be morally responsible can be interpreted as
being an appropriate object of vicarious reactive attitudes
(\cite[p.3]{SmithVickers2021}) (\cite[p.175]{Matthias_2004}). Tigard takes moral
responsibility in this regard \say{as a social funtion of [...] reactive
attitudes} (\cite[p.3]{Tigard_2020}).  FIND MORE INTERPRETATIONS AND PUT THEM
INTO CONTEXT

SHOEMAKER SAYS: MORAL RESPONSIBILITY IS, AT LEAST IN PART, BEING OPEN TO A
CERATAIN RANGE OF MORAL RESPONSES (PAGE 11)


Before moving on, Strawson, introduces another idea, which might be important for
our further discussion on the responsibility gap. He describes in which cases
reactive attitudes are mitigated or even not exhibited at all. Strawson
distinguishes two general groups of such cases:

\begin{enumerate}
	\item Cases of the first group are those where the source of injury is a
		moral agent but their explanation for their action can be
		summarised with the sentences `I didn't know', `I had to do it'
		or something similar (\cite[p.7-8]{Strawson1962}). Tigard
		descibes these cases as situations \say{where the agent is
		normal, but the circumstances are abnormal [...]} 
		(\cite[p.5]{Tigard_2020}).

		Examples of such cases are the
		gentleman who accidentally steps on someones foot because the train is too
		full and he tries to navigate through the crowd, or the doctor
		who has lost a patient and is then rude to her husband. The
		people who suffer the injury usually tend to modify their
		reactive attitudes to fit the circumstances.
	\item The second group is again nicely described by Tigard as cases
		\say{where the circumstances are normal but the agent is
		abnormal} (\cite[p.5]{Tigard_2020}). Strawson speaks of children
		or schizophrenics or people that act out of compulsion. Such
		agents cannot be appropriate targets of reactive attitudes
		because the expectations upon which the reactive attitudes are
		based cannot be reasonably targeted towards them. According to
		Strawson, it is unreasonable to expect moral behaviour from
		someone who is morally deranged or underdeveloped. In this
		sense, they are not moral agents and cannot be treated as such.
		We do not see them as members of the moral community
		(\cite[p.18]{Strawson1962}). The attitudes we exhibit towards
		them differ accordingly compared to those who are members of the
		moral community. We see them as \say{object[s] of social policy;
			as [...] subject[s] for [...] treatment; as something
			certainly to be taken account, perhaps precautionary
			account of; to be managed or handled or cured or
		trained; perhaps simply to be avoided [...]}
		(\cite[p.8]{Strawson1962}). Seeing an agent as such, implies
		that we portray a second set of attitudes towards them. Strawson
		calls these attitudes \textit{objective attitudes}
		(\cite[p.9]{Strawson1962}). 
\end{enumerate}

I want to reiterate: To have reactive attitudes towards someone \textit{means}
to view them as a fully responsible agent. In Strawson's eyes these are the same
things (\cite[p.23]{Strawson1962}). To have objective attitudes towards someone (or something)
\textit{means} to view them outside of the moral community and, thus, to view
them as an inadequate target for ascribing responsibility. \#foreshadowing\\

\subsection{Shoemaker}

 I already have repeadetly used such phrasings as `inadequate target' or
 `appropriate object' of moral responsibility or reactive attitudes or blame or
 praise. However, the attentive reader will find that Strawson's account of our
 moral practices focuses strongly on our external perceptions of other's
 internal attitudes. In this regard, we are prone to say, that someone is an
 appropriate object of blame, if (1) we see them as a member of the moral
 community (we can develop reactive attitudes towards them) and (2) we
 \textit{interpret} their attitudes as malevolent or indifferent towards us. But what
 about the cases where our interpretation is wrong? We might think their action
 is an expression of ill will towards us, but by looking beneath their sculp, we
 might see that it is actually not the case and we had misinterpreted their
 attitude. Intuitively, it would not be fair to blame someone, if their \textit{real}
 attitude would not correspond with our \textit{perception} of their attitude.
 Or their action was subject to circumstances unbeknownst to us. We hold them
 responsible and blame them for the action. But upon learning more about the
 circumstances we change our mind and judge the person to be not responsible
 anymore. In fact, we say that they have not been responsible at all even for the
 time we thought they were responsible. Does this not show that there is a sense
 of being responsible that goes beyond `being held responsible' by others? Does
 this not show that we in general do believe in a kind of responsibility relies
 less on our perception and more on the truth of the situation? And it would not
 be fair for us to hold someone responsible, if they `in reality' are not
 responsible (\cite[p. 472]{Smith_2007}).\\
 \textit{Oh hey what is that? I think that is determinism creeping in the
	 background! Oh no, I hope nobody will notice it! *MISCHA USES HIS
 SPECIAL ATTACK: \textbf{GLOSSING OVER STUFF THAT MIGHT BE VERY IMPORTANT}*}\\

 \textit{Can you feel it's presence? It is still at distance, yet comes closer the more we
 seek for truth. It regards us patiently as it is the end boss and it waits for
 a fight that may never halt. And the fight might never halt; not because we are
 equal adversaries, but because we are too weak to know when it is over.}

 \textit{It's breath is cold and merciless. It knows. But as the great minds, whose work
 is the basis of the words you read, I shall continue to ignore this final
 question and move on. And I ask the same of you. With the words of the russian
 writer Mikhail Bulgakov: Follow me, reader!}

 I believe, Strawson would answer to this, that his account of our moral practices
 does not claim to be fair or fulfill all the demands we might have towards said
 practices it is not even necessarily internally consistent. In this sense, it is
 not normative but largely descriptive. In practice, when blaming someone, we do
 not distinguish between our percetion of their attitude and their \textit{real}
 one.
 THIS IS SOMETHING THAT I WANT TO SAY, BUT BETTER:
 Being a descriptive account of our moral practices, we can take it as it is
 standing alone, but it is also capable of leaving enough space to place a
 normative theory inside of it to complement and it. Shoemakers account of
 responsibility is such a theory, that can be placed into strawsons account.

 Angela Smith mentions the difference between being held responsible and being
 actually responsible

 Still, we find other philosophers seeking for a more rigorous and satisfying
 account of `appropriate' in this context.  We have encountered one such approach
 in Matthias' control requirement (\cite[p.175]{Matthias_2004}), which we will
 dicuss in a later section. @FUTURE\_MISCHA: HAVE YOU REALLY DISCUSSED IT?  DO
 YOU HAVE ENOUGH MATERIAL TO DISCUSS IT?
 I also want to introduce an additional approach, proposed by David Shoemaker
 (\cite{Shoemaker_2011}). DA SHOE tries to give an account of what it means to
 be truly responsible. He proposes three distinct types of responsibility:
 Answerability, Attributability and Accountability.

 Fro mthis it is easy to interpret that moral responsibility
 rests solely on other's impression of one's will (though this is not a
 necessary conclusion or even something that Strawson argues for). In this regard, we might
 want to say that this seems too one-sided. If Strawsons account depends so
 heavily on attitudes, it shall not neglect the blamee's `true' attitudes.

 based on our perception of
 their attitude, if thi


According to Strawson, Moral Responsibility must not be a metaphysical entity,
but rather manifests as a result of human nature and our social practices.




With this in mind, instead of asking 'What is moral responsibility?', the better (and
certainly easier) question to ask is: What does it mean to be morally responsible?

In ``Freedom and Resentment'' P.F.Strawson explains that reactive attitudes. Bla
bla We expect a certain behaviour from other people and depending on wether they
cohere with these expectations we exhibit resentment or gratitude towards their
behaviour[Mehr ins detail gehen]. Some philosophers say that responsibility is
the property that allows us to appropiately target an agent with gratitude or
resentment for something they have done [zitieren/umschreiben das sind nur
dreckige sätze]. Bla bla.

% Das ist doch mal ne schöne formulierung!
In light of Strawsons refusal to see moral responsibility as a metaphysical
entity, 'What is moral responsibility?' is perhaps the wrong question to ask.
The better (and certainly easier) question is: What does it mean to be morally
responsible (for something)?

So what are the cases, in which we appropriately say that an agent is
responsible. Explain CR again. Explain when an agent is excused and when they
are exempted from being held responsible.

Extend the model of responsibility to shoemakers 'accountability, answerability
and explainability model'



When we talk about moral responsibility, we must probably first explore what we
mean by that term. Specifically we need to answer two central questions:\\

\begin{enumerate}
	\item In which cases can somebody be held responsibly?
	\item What does moral responsibility entail?
\end{enumerate}
%The answer to the second question can easily be regarded as a functional
%definition of moral responsibility.
For the sake of a focused and productive argumentation I will, for the duration
of this entire work, assume that the concept of moral responisibility is
important and is necessary for a functioning and ethical society
(\acrshort{mrinfes}) without
providing an argument for this assumption. Questioning this assumption would, I
believe, fill a whole other bachelor's thesis and likely even more. In the sense of
this assumption, I will also ignore the debate around free will and how it is
connected to moral responisbility.

The Control Requirement
More complex models of responsibility

Moral Agency
\section{Can we Bridge the Gap}

Who are the candidates?: The manufacturer, the user, the machine
If the machine is responsible does it imply moral agency/ we must develop
reactive attitudes. -> What are the conditions for developing reactive attitudes
towars machines (Statistically responsible AI Vickers and Smith)
Shoemaker has another example about the aliens, that might eb fitting here

Essentially: How do machines fit into these frameworks
Upper bound - lower bound of moral agents

Yes we can: Here is how
Instrumentalism 2.0
	There is a moral risk in using unpredictable machines and the
	users/manufacturers that use them accept this risk and are (implicitly)
	accepting the responsibility. Analogy: There is a risk in using medical
	drugs because of the side effects.
Machine Ethics
Hybrid responisibility

No, we cant: Here is why:

\section{Real World Problems}
\subsection{Autonomous Weapon Systems}
\subsection{Healthcare}
\subsection{COMPAS}
\section{Discussion}
\section{Conclusion}
\section{Disclaimers}
Put this in the beginning or even better into the introduction

Blame and praise are asymmetrical in how we pay attentio to them . While it
might be an interesting intellectual execise to think about who deserves credit
for a piece of art produced by a machine learning algorithm the question of
responsiblity seems far more pressing for when an automated car runs over a
pedestrian or a medical software misdiagnoses a patient. I will thus, mostly
restrict my search for responsibility to cases in which we want to blame.
REFERENCE: SMITH BEING RESPONSIBLE VS HOLDING RESPIONSIBLE P.5
\section{Acknowledgements}
\clearpage

\printglossary[type=\acronymtype]
\printglossary
\printbibliography
\end{document}
