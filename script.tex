\documentclass{article}
\usepackage[utf8]{inputenc}
\usepackage{dirtytalk}
\usepackage{color}
\usepackage[acronym]{glossaries}
\usepackage{biblatex}
\addbibresource{bib.bib}
\title{Working title}
\author{Mischa}


\makeglossaries

\newacronym{mrinfes}{mRINFES}{Moral Responsibility is Important and Necessary for a Functional and Ethical Society}
\newacronym{cr}{CR}{Control Requirement}
\newacronym{ai}{AI}{Artificial Intelligence}
\newacronym{ml}{ML}{Machine Learning}
\newacronym{aws}{AWS}{Autonomous Weapon System}
\begin{document}
\begin{titlepage}
	\maketitle
\end{titlepage}
\tableofcontents
\newpage
\section{Introduction}

In this work I will discuss technology and moral responsibility and how the two
relate to each other. Specifically, I will investigate the different ways we can seek
for moral responisbility in situations where an autonomous machine is involved.

Put more here.

But first, let us examine the traditional way of how we ascribe moral
responisbility in situations where technology is involved:

Suppose the following situations: A person hits another person with a hammer and
kills them. A newly installed dam breaks and a city is flooded. A hacker manages
to get access to a digital banking system through his own computer and steals a good deal of money.\\

The hammer, the dam and the hacker's computer are technology that is directly
involved in morally critical situations. Yet, we abstain from blaming these
artifacts for what has happened in the respective situations. We also do not put
the events off as natural tragedies, as we do when a storm destroys a house or
an avalanche kills a skier in the mountains. We naturally ascribe the
responsibility for the events to the people behind the technology.
The person who wielded the hammer, the architect of the dam, the hacker.
These people used the technology as a tool to achive their own end and they are
responsible for the effects, that the technology has on our world, whether they
achieve the end or not (in the case of the dam-architect).
To view technology as tools or instruments used by humans and the humans as the ultimately
responsible entities for the technology is what Heidegger calls the \textit{instrumentalist
definition of technology} \cite{heidegger1977technology}. 

\say{We ask the question concerning technology when we ask what it is. Everyone
knows the two statements that answer our question. One says: Technology is a means to an end. The
other says: Technology is a human activity.}\cite[p.4]{heidegger1977technology}

Thus, according to the instrumentalist definition, technology is something that
is intimately connected to humans and inherits its moral standing from the
quality of the \textit{human} end that it tries to achieve and from the way
it is used by \textit{humans}. Technology is  not moral on its own. Only in the
context of human ends and actions. If it is humans that decide to use technlogy
for some end it is naturally them, who are responsible for the results.\\

In the advent of machine learning we are facing a type of technology that is
intentinally becoming more and more autonomous, making decisions on its own
without any human supervision, without any human being able to predict the
decisions and even without any human being able to explain the decisions. The
Machines are essentially black boxes making decisions and affecting the world in
morally significant ways:

Autonomous Vehicles are being developed to populate the streets and naviagte
dangerous situations.
Machine Learning Algorithms analyse our behaviour on the internet, recommend
content that they find we would be interested in and influence us in this
manner.
There is a multitude of applications for ML in health care. We can easily
imagine that medical practitioners will increasingly rely on tools that diagnose
diseases and even propose treatments. Eventually, even patients might even cut
out the middle man and receive their medical care from artificial physicians.
There already are services that provide a kind of psychotherapy by texting with
a chatbot. [zitieren]
Autonomous Weapon Systems (\acrshort{aws}) are being developed. The aim is to
create war robots that can be sent into the battle field and they would be able
to decide on their own whether to kill a target or not.



%Instrumental Theory:

%\cite{johnson2006computer}: Computers are moral entites but not moral agents
%For most of the time human technology has been used as a tool.
%The user or manufacturer has the responisibility for what the tool does.
%This has been working so far quite well and the instrumental theory was a valid
%and accepted extension of our moral understanding.
%We are now in a time, where new technologies, that exhibit more and more
%autonomy challenge this stance, that they are mere tools. How should we handle
%des question of responisibility with this new technology? Is there are
%responisibility gap?
%
%Introduction of new technologies: ML

\subsection{The Dirty Problem}

According to Matthias \cite{Matthias_2004} the reason why we can hold either the
manufacturer or the operator of a machine responsible for what it effects in the
world, is because we can sensibly say that they are the moral agents who were in
control of said machine. Matthias claims that responsibility implies control:

\vspace{.8em}
\say{[An] agent can be considered responsible only if he  knows the particular
facts surrounding his action, and if he is able to freely form a decision to
act, and to select one of a suitable set of available alternative actions based
on these facts.} \cite[p.175]{Matthias_2004}

\vspace{.8em}
This means that we can only hold someone responsible for something they have
done if they had sufficient control over their action.
This is widely refered to as the Control Requirement (\acrshort{cr}) [zitieren].

Converesly, if an agent does not have sufficient control, we can acsribe at most
partial responsibility, if any, to them.\\
This notion of control complements the intrumental theroy nicely: The
manufacturer and operator have control over their machines, thus they are the
ones who are responsible if something happens because of the machines.
It is then very clear how to assign responisbility in critical situations: If the operator uses the
machine in accordance with the manufactureres specifications and something goes
wrong, we say that the manufacturer is responsible. If the operator deviates
from the manufacturers specifications and something goes wrong, we say the
operator is responsible.

Enter Machine Learning:
We are now in a time where computer scientists and engineers work on increasing
the autonomy of their programs and machines by using techniques that can be
bundeled by the term \textit{machine learning} \acrshort{ml}. `Autonomy' in this sense
means that we allow the technology to make it's own decisions based on it's
prior experience without the programmer or user knowing what the decision is
going to be. 

%Put here: Machines can play Go, AWS, medical exper systems, even the youtube
%algorithm

This leads to a change in the classical roles of the manufacturer
(programmer) and operator (user) insofar that the amount of control they exert
over the machines diminishes as the machines become more and more autonomous.
Arguably, our traditional ways of ascribing responsibility are challenged. If we
agree with the CR and 

%Current practices in Artificial Intelligence (\acrshort{ai}) challenge this
%stance. The classical roles of manufacturer (programmer) and operator (user)
%change insofar that the amount of control they exert over the machines
%diminishes as the machines become increasingly autonomous.

Current practices in Artificial Intelligence (\acrshort{ai}) aim at increasing
the autonomy of machines and their software [zitieren]. `Autonomy' in this sense
means that we allow the technology to make it's own decisions based on it's
prior experience without the programmer or user knowing what the decision is
going to be. In other words, they reduce the amount of control the programmer or
user has over the machines. This is primarily the case for technology that uses
machine learning to acquire a certain behaviour. [Passt hier eine beschreibung
von machine learning rein?] At the same time the actions of
such technologies have more and more impact on the people surrounding them
[zitieren].\\

The real question:
Now here comes the question of this work: Who can sensibly be held responsible for the
actions of an autonomous machine?
Who is responsible if a driverless car runs over a person? Who is responsible if
an artificial physician proposes a wrong treatment for a patient? Who is
responisble for a war crime commited by an autonomous weapon system
(\acrshort{aws})?\footnote{Talk about asymetry of blame and reward. Refer to
later in next section perhaps, because here I mention responisbility only in the
sense of blaming someone}\\

Matthias says that our current practices of ascribing moral responsibility
fail at finding an appropriate target when an autonomous machine is
strongly involved in a situation. He calls this problem \textit{the
responsibility gap}.\\
%are not designed for dealing with autonomous machines and we are, thus, facing
%a responsibility gap. \\

The considerations above, do not entail that our current practices of
dealing with the effects machines have on our world must necessarily change,
but rather that the contemporary and foreseeable developments in AI and ML
challenge our current practices and motivate their reevaluation.

On the following pages I will first give descriptions of various accounts for
moral responsiblity and will then proceed to describing and debating the
different approaches philosophers have proposed for dealing with the assumed
responisbility gap.

%BIS HIER
%
%The idea that manufacturers and users lose control over an autonomous machine
%does not only stem from the theoretical definition of the word 'autonomous'. As
%Matthias argues, the loss of control can be found in the very way we put
%ML-techniques into practice \cite[p.181-182]{Matthias_2004}. 
%
%Russell and Norvig define an agent as learning 
%
%\vspace{.8em}
%\say{if it improves its performance on future taskes after making observations
%about the real world.}\cite[p.693]{russell2010artificial}
%
%\vspace{.8em}
%
%Technology that is being developed today often has the aim to become more and
%more autonomous. Autonomous in this sense means that we allow the technology to
%make it's own decisions based on it's prior experience without the programmer or
%user knowing what the decision is going to be. This is primarily the case for
%technology that uses machine learning to acquire a certain behaviour.
%As Matthias \cite{Matthias_2004} points out, the growing autonomy challenges our
%existing pracices for ascription of moral responsibility. Matthias says that
%in order to be responsible for something, one requires to be in control of it.
%This widely refered to as the Control Requirement (\acrshort{cr}) [zitieren].
%It follows, that neither the programmer, nor the user of an autonomous machine,
%can sensibly be said to be responsible for the actions of that machine.

\section{What is Moral Responsibility}

Before jumping into any analysis of the responsibility gap as described above,
it makes sense to first explore what I mean, when I speak of moral
responsibilty.

There is an ongoing discussion in philosophy about the existance of determinism
and it's impact on free will, wich in turn appears to be closely related to
moral responsibility [zitieren]. To discuss this topic is notbla bla nor is it my intention to take a position on
this issue. Instead I will try to elegantly sidestep the matter by taking a Strawsonian approach
on moral responsibility.

In ``Freedom and Resentment'' P.F.Strawson explains that reactive attitudes. Bla
bla We expect a certain behaviour from other people and depending on wether they
cohere with these expectations we exhibit resentment or gratitude towards their
behaviour[Mehr ins detail gehen]. Some philosophers say that responsibility is the property that allows
us to appropiately target an agent with gratitude or resentment for something
they have done [zitieren/umschreiben das sind nur dreckige sätze]. Bla bla.



When we talk about moral responsibility, we must probably first explore what we
mean by that term. Specifically we need to answer two central questions:\\

\begin{enumerate}
	\item In which cases can somebody be held responsibly?
	\item What does moral responsibility entail?
\end{enumerate}
%The answer to the second question can easily be regarded as a functional
%definition of moral responsibility.
For the sake of a focused and productive argumentation I will, for the duration
of this entire work, assume that the concept of moral responisibility is
important and is necessary for a functioning and ethical society
(\acrshort{mrinfes}) without
providing an argument for this assumption. Questioning this assumption would, I
believe, fill a whole other bachelor's thesis and likely even more. In the sense of
this assumption, I will also ignore the debate around free will and how it is
connected to moral responisbility.

The Control Requirement
More complex models of responsibility

Moral Agency
\section{Can we Bridge the Gap}

Who are the candidates?: The manufacturer, the user, the machine
If the machine is responsible does it imply moral agency/ we must develop
reactive attitudes. -> What are the conditions for developing reactive attitudes
towars machines (Statistically responsible AI Vickers and Smith)

Essentially: How do machines fit into these frameworks
Upper bound - lower bound of moral agents

Yes we can: Here is how
Instrumentalism 2.0
	There is a moral risk in using unpredictable machines and the
	users/manufacturers that use them accept this risk and are (implicitly)
	accepting the responsibility. Analogy: There is a risk in using medical
	drugs because of the side effects.
Machine Ethics
Hybrid responisibility

No, we cant: Here is why:

\section{Real World Problems}
\subsection{Autonomous Weapon Systems}
\subsection{Healthcare}
\subsection{COMPAS}
\section{Discussion}
\section{Conclusion}
\section{Acknowledgements}
\clearpage

\printglossary[type=\acronymtype]
\printglossary
\printbibliography
\end{document}
